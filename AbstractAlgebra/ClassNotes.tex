\documentclass[11pt]{article}
\pagestyle{empty}

\setlength{\topmargin}{-.75in}
\setlength{\textheight}{9.2in}
\setlength{\oddsidemargin}{-.25in}
\setlength{\evensidemargin}{-.25in}
\setlength{\textwidth}{6.75in}
\setlength{\parskip}{4pt}

\usepackage{amsfonts}
\usepackage{amsthm}
\documentclass{amsart}
\usepackage[english]{babel}
\usepackage[utf8]{inputenc}
\usepackage{amssymb}
\usepackage{mathtools}
\usepackage{amsmath}
\usepackage{breqn}
\setlength{\textfloatsep}{5pt}
\newtheorem{thm}{Theorem}[section]
\newtheorem{prop}[thm]{Proposition}
\newtheorem{lem}[thm]{Lemma}
\newtheorem{cor}[thm]{Corollary}
\theoremstyle{definition}
\newtheorem{definition}[thm]{Definition}
\newtheorem{example}[thm]{Example}
\newtheorem{dis}[thm]{Discussion}
\newtheorem{rem}[thm]{remark}
\newcounter{casecount}
\setcounter{casecount}{0}
\newenvironment{case}{\refstepcounter{casecount}\textbf{Case \arabic{casecount}:}}{}

\usepackage{etoolbox}
\AtBeginEnvironment{proof}{\setcounter{casecount}{0}}
\newtheorem{remark}[thm]{Remark}
\numberwithin{equation}{section}
\newcommand{\R}{\mathbb{R}}  % The real numbers.
\newcommand{\Q}{\mathbb{Q}}  % The rational numbers.
\newcommand{\C}{\mathbb{C}}  % The rational numbers.
\newcommand{\Z}{\mathbb{Z}}
\newcommand{\D}{\mathcal{O}_K}
\newcommand{\p}{\mathfrak{p}}
\newcommand{\B}{\mathfrak{B}}
\DeclareMathOperator{\dist}{dist} % The distance.

\def\Out{\mathrm{Out}}
\begin{document}

\begin{center}
{\sf\LARGE Class February 23}
\end{center}

\begin{enumerate} %I will number each of the proofs that I perform using this.

\item Direct Products and Finite Abelian Groups
\begin{definition}
$G$ always denotes a group. G is the inner direct product of the subgroups $A, B \leq G$ if i) $A \vartriangleleft G$, $B \vartriangleleft G$ ii) $A \cap B = \{e\}$ iii) $G = AB$. The notation for direction products is $G = A \times B$.
\end{definition}
\begin{lem}
Assume $G = A \times B$. \\
(a) $A$ and $B$ commute element-wise i.e. $ab = ba$ $\forall a \in A, b \in B$.\\
(b) if $A$ and $B$ are abelian, then so is $G$.
\end{lem}
\begin{proof}
(a) Consider the commutators $[a, b] := (aba^{-1})b^{-1} = a(ba^{-1}b^{-1}) \in A \cap B = \{e\}$\\
$\implies aba^{-1}b^{-1} = e \implies ab = ba$ $\forall a \in A, b \in B$\\
(b) $g_1, g_2 \in G \implies \exists a_1, a_2 \in A, b_1, b_2 \in B$ s.t. $g_1 = a_{1}b_{1}$ and $g_2 = a_{2}b_{2} \implies g_{1}g_{2} = a_{1}b_{1}a_{2}b_{2} = a_{1}a_{2}b_{1}b_{2}$ and because $A, B$ are abelian, this equals $a_{2}(a_{1}b_{2})b_{1} = a_{2}b_{2}a_{1}b_{1} = g_{2}g_{1}$
\end{proof}
\begin{example}
(a) $V = <(12)(34)> \times <(13)(24)> \cong \Z_2 \times \Z_2 $\\
(b) $U(8) = \{[1], [3], [5], [7]\} = <[3]> \times <[5]> \cong \Z_2 \times \Z_2$.\\
(c) $\Z_6 = <[3]> \cong \Z_3 \times <[2]> \cong \Z_2$\\
(d) $D_6 = \{b^{i}, ab^{i} \shortmid 0 \leq i \leq 5 \}$ such that $a^{2} = b^{6} = e$, $aba^{-1} = aba = b^{-1}$. Therefore, $D_6 \cong <b^{3}> \times \{e, b^{2}, b^{4}, a, ab^{2}, ab^{4}\}$ $\implies D_6 \cong \Z_2 \times D_3$\\
(e) By contrast, neither $D_4$ nor $Q_8$ can be written as direct products of two proper subgroups (Exercise).\\
(f) Trivially, $\forall G, G = G \times \{e\}$
\end{example}
\begin{lem}
If $|G| = p^{2}$, $p$ is prime, then either $G$ is cyclic or $G = A \times B (\cong \Z_p \times \Z_p)$ with subgroups $A$ and $B$ of order $p$.
\end{lem}
\end{enumerate}

\begin{center}
{\sf\LARGE Class February 26}
\end{center}

\end{document}
