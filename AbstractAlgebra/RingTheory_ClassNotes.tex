\documentclass[11pt]{article}
\pagestyle{empty}

\setlength{\topmargin}{-.75in}
\setlength{\textheight}{9.2in}
\setlength{\oddsidemargin}{-.25in}
\setlength{\evensidemargin}{-.25in}
\setlength{\textwidth}{6.75in}
\setlength{\parskip}{4pt}

\usepackage{amsfonts}
\usepackage{amsthm}
\documentclass{amsart}
\usepackage[english]{babel}
\usepackage[utf8]{inputenc}
\usepackage{amssymb}
\usepackage{mathtools}
\usepackage{amsmath}
\usepackage{breqn}
\usepackage{dsfont} %for the Hamilton's quaternions
\setlength{\textfloatsep}{5pt}
\newtheorem{thm}{Theorem}[section]
\newtheorem{prop}[thm]{Proposition}
\newtheorem{lem}[thm]{Lemma}
\newtheorem{cor}[thm]{Corollary}
\theoremstyle{definition}
\newtheorem{definition}[thm]{Definition}
\newtheorem{example}[thm]{Example}
\newtheorem{dis}[thm]{Discussion}
\newtheorem{rem}[thm]{Remark}
\newcounter{casecount}
\setcounter{casecount}{0}
\newenvironment{case}{\refstepcounter{casecount}\textbf{Case \arabic{casecount}:}}{}

\usepackage{etoolbox}
\AtBeginEnvironment{proof}{\setcounter{casecount}{0}}
\newtheorem{remark}[thm]{Remark}
\numberwithin{equation}{section}
\newcommand{\R}{\mathbb{R}}  % The real numbers.
\newcommand{\Q}{\mathbb{Q}}  % The rational numbers.
\newcommand{\C}{\mathbb{C}}  % The rational numbers.
\newcommand{\Z}{\mathbb{Z}}
\newcommand{\N}{\mathbb{N}} %the natural numbers
\newcommand{\D}{\mathcal{O}_K}
\newcommand{\p}{\mathfrak{p}}
\newcommand{\B}{\mathfrak{B}}
\DeclareMathOperator{\dist}{dist} % The distance.

\def\Out{\mathrm{Out}}
\begin{document}

\begin{center}
{\sf\LARGE Some Ring Theory Class Notes}
\end{center}

\begin{center}
{\sf\LARGE Class March 12}
\end{center}

Conventions regarding 1 (multiplicative unity):
\begin{enumerate}
    \item Every ring $R$ has a multiplicatuve unity denoted by $1$ or $1_R$ such that $1*a = a*1$ $\forall$ $a \in R$. Note: $1 = 0$ in $R$ $\Leftrightarrow$ $R = \{0\}$ because $\forall a \in R$: $a = a * 1 = a * 0 = 0$.
    \item Any subring $S$ of $R$ must contain $1_R$. For subring, check
    \begin{enumerate}
        \item $1_{R} \in S$
        \item $a \in S$ $\implies$ $-a \in S$
        \item $a, b \in S$ $\implies$ $a + b \in S$
        \item $a, b \in S$ $\implies$ $ab \in S$
    \end{enumerate}
    Note: An ideal $I$ of $R$ is a subring if and only if $I = R$ ($1 \in I$ $\implies$ $a = a*1 \in I$ $\forall$ $a \in R$).\\
    \begin{example}
    $R \times \{0\} = \{(a, 0) \mid a \in R\}$ is not a subring of $R \times R$ if $R \neq \{0\}$ since $(1, 1) \notin R \times \{0\}$. But $\{(a, a) \mid a \in R\}$ is a subring of $R \times R$.
    \end{example}
    \item For any ring homomorphism $\varphi: R \rightarrow S$ we require $\varphi(1_R) = 1_S$. Note that this is not a consequence of the other ring homomorphism properties:
    \begin{enumerate}
        \item $\varphi(a + b) = \varphi(a) + \varphi(b)$ $\forall a, b \in R$
        \item $\varphi(ab) = \varphi(a)\varphi(b)$ $\forall$ $a, b \in R$
    \end{enumerate}
    $\varphi(0) = 0$ is a consequence of $(a)$: $\varphi(0) = \varphi(0 + 0) = \varphi(0) + \varphi(0)$ $\implies $ $0 = \varphi(0)$. For multiplication, $\varphi(1) = \varphi(1*1) = \varphi(1)*\varphi(1)$ does not necessarily imply $1 = \varphi(1)$ since $\varphi(1)$ need not have a multiplicative inverse in $S$.
    \begin{example}
    $\varphi: R \rightarrow R \times R$ which maps $a \rightarrow (a, 0)$ is NOT a ring homomorphism since $\varphi(1_R) = (1_R, 0) \neq 1_{R \times R}$ if $R \neq \{0\}$
    \end{example}
    \begin{example}
    $\psi: R \rightarrow R \times R$ which maps $a \rightarrow (a, a)$ is a ring homomorphism.
    \end{example}
    \item For an integral domain $R$ (commutative without zero divisors) we also require $1 \neq 0$ $\Leftrightarrow$ $R \neq \{0\}$ (neither integral domain nor a field)
    \begin{example}
    \begin{enumerate}
        \item of fields: $\R, \Z_p$ ($p$ prime), $\Q, \C$. $\Q(\sqrt{2}) := \{a + b\sqrt{2} \mid a, b \in \Q\}$ subfield of $\R$. Check: $0 \neq x \in \Q(\sqrt{2})$ $\implies$ $x^{-1} \in \Q(\sqrt{2})$ (need $\sqrt{2} \notin \Q$).
        \item of integral domains which are no fields: $\Z$, when $n$ is a prime $\implies$ $\Z_n$ is an integral domain, but also a field. When $n$ is not a prime $\implies$ $\Z_n$ has zero divisors and isn't an integral domain. Specifically $\exists$ $l, m \in \N$, $1 < l, m < n$ such that $n = lm$ $\leadsto$ (modulo $n$). $[0] = [n] = [lm] = [l][m]$ in $\Z_n$ (such that $[l] \neq [0]$ and $[m] \neq [0]$.
        \item $\Z[i] = \{a + bi \mid a, b \in \Z\}$ subring of $\C$; $\Z[\sqrt{2}]$ is a subring of $\R$.
        \item commutative rings which are not integral domains. $\Z_n$, $n$ is not prime. $\Z \times \Z$ has zero divisors e.g. $(1, 0)*(0, 1) = (0, 0)$.
        \item of non-commutative rings:
        \begin{enumerate}
            \item $M(n, R)$, $n \geq 2$ and $R$ any ring $\neq \{0\}$. $\exists A, B \in M(n, R)$ such that $AB \neq BA$
            \item Hamilton's quaternions
            $\mathds{H} = \{a+bi+cj+dk \mid a, b, c, d \in \R\}$ ($\cong \R^{4}$ as abelian group). Multiplication is induced by that $\Q$ and distributive laws $\leadsto$ example of skew field or division ring.
        \end{enumerate}
    \end{enumerate}
    \end{example}
\end{enumerate}
\end{document}
